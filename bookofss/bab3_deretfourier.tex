\chapter{Deret Fourier Fungsi Periodik}

\section{Pengantar}
Fungsi periodik banyak ditemui dalam berbagai masalah teknik. Fungsi tersebut biasanya lebih rumit daripada fungsi sinus dan kosinus. Deret Fourier merupakan upaya untuk menyatakan sembarang fungsi periodik kedalam fungsi sinus dan kosinus.

Suatu fungsi dinyatakan periodik jika fungsi tersebut terdefinisikan untuk semua $x$ real dan jika terdapat bilangan positif T sedemikian hingga

\begin{equation}
\label{sec:periodik}
f(x+T)=f(x),  
\end{equation}
untuk semua $x$.

Bilangan T disebut periode dari fungsi $f(x)$. Menurut persamaan~\ref{sec:periodik} jika $n$ bilangan bulat  sembarang, maka berlaku

Konsep sinyal dan sistem muncul dalam dalam berbagai aplikasi yang berbeda bergantung pada bagaimana pengunaan konsep tersebut. Konsep sinyal dan sistem dapat digunakan untuk mengkaraterisasi sebuah sistem. Mengkarakterisasi berarti membaca karakter atau sifat dari sebuah sistem tertentu dengan cara memberikan sinyal masukan yang bervariasi dan membaca bagaimana sistem menanggapai masukan yang bervariasi tersebut. Hal ini seperti apabila kita mempunyai sebuah kotak hitam yang kita tidak tahu sepeti apa sifat dan bagaimana kotak hitam tersebut bekerja. Dengan memberikan masukan bervariasi terhadap kotak hitam tersebut maka akan dapat diperoleh keluaran barupa tanggapan yang berbeda dengan masukan, sehingga dengan ini kita dapat mengetahui apa dan bagaimana kotak hitam misterius ini berperilaku atau bekerja. Konsep sinyal dan sistem juga dapat digunakan untuk melakukan pemrosesan terhadap sinyal tertentu agar diperoleh keluaran sinyal sesuai yang diharapkan atau untuk menghilangkan sinyal yang tidak diinginkan. Contohnya adalah dalam sistem \textit{noise canceling} (NC) atau pembatal bising. Dalam sistem NC yang biasa digunakan pada speaker, bising atau suara mengganggu yang tidak diinginkan dari dapat diredam sedemikian hinga sehingga suara atau bunyi yang keluar dari \textit{speaker} adalah ahanya suara yang diharapkan. Sistem NC akan membangkitkan suara dengan frekuensi tertentu yang sama dengan suara bising tertentu tetapai dengan fase yang berbeda sehingga melalui proses interferensi suara bising dapat diredam. 

